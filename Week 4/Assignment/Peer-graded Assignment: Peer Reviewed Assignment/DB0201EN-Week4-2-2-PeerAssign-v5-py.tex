\documentclass[11pt]{article}

    \usepackage[breakable]{tcolorbox}
    \usepackage{parskip} % Stop auto-indenting (to mimic markdown behaviour)
    
    \usepackage{iftex}
    \ifPDFTeX
    	\usepackage[T1]{fontenc}
    	\usepackage{mathpazo}
    \else
    	\usepackage{fontspec}
    \fi

    % Basic figure setup, for now with no caption control since it's done
    % automatically by Pandoc (which extracts ![](path) syntax from Markdown).
    \usepackage{graphicx}
    % Maintain compatibility with old templates. Remove in nbconvert 6.0
    \let\Oldincludegraphics\includegraphics
    % Ensure that by default, figures have no caption (until we provide a
    % proper Figure object with a Caption API and a way to capture that
    % in the conversion process - todo).
    \usepackage{caption}
    \DeclareCaptionFormat{nocaption}{}
    \captionsetup{format=nocaption,aboveskip=0pt,belowskip=0pt}

    \usepackage[Export]{adjustbox} % Used to constrain images to a maximum size
    \adjustboxset{max size={0.9\linewidth}{0.9\paperheight}}
    \usepackage{float}
    \floatplacement{figure}{H} % forces figures to be placed at the correct location
    \usepackage{xcolor} % Allow colors to be defined
    \usepackage{enumerate} % Needed for markdown enumerations to work
    \usepackage{geometry} % Used to adjust the document margins
    \usepackage{amsmath} % Equations
    \usepackage{amssymb} % Equations
    \usepackage{textcomp} % defines textquotesingle
    % Hack from http://tex.stackexchange.com/a/47451/13684:
    \AtBeginDocument{%
        \def\PYZsq{\textquotesingle}% Upright quotes in Pygmentized code
    }
    \usepackage{upquote} % Upright quotes for verbatim code
    \usepackage{eurosym} % defines \euro
    \usepackage[mathletters]{ucs} % Extended unicode (utf-8) support
    \usepackage{fancyvrb} % verbatim replacement that allows latex
    \usepackage{grffile} % extends the file name processing of package graphics 
                         % to support a larger range
    \makeatletter % fix for grffile with XeLaTeX
    \def\Gread@@xetex#1{%
      \IfFileExists{"\Gin@base".bb}%
      {\Gread@eps{\Gin@base.bb}}%
      {\Gread@@xetex@aux#1}%
    }
    \makeatother

    % The hyperref package gives us a pdf with properly built
    % internal navigation ('pdf bookmarks' for the table of contents,
    % internal cross-reference links, web links for URLs, etc.)
    \usepackage{hyperref}
    % The default LaTeX title has an obnoxious amount of whitespace. By default,
    % titling removes some of it. It also provides customization options.
    \usepackage{titling}
    \usepackage{longtable} % longtable support required by pandoc >1.10
    \usepackage{booktabs}  % table support for pandoc > 1.12.2
    \usepackage[inline]{enumitem} % IRkernel/repr support (it uses the enumerate* environment)
    \usepackage[normalem]{ulem} % ulem is needed to support strikethroughs (\sout)
                                % normalem makes italics be italics, not underlines
    \usepackage{mathrsfs}
    

    
    % Colors for the hyperref package
    \definecolor{urlcolor}{rgb}{0,.145,.698}
    \definecolor{linkcolor}{rgb}{.71,0.21,0.01}
    \definecolor{citecolor}{rgb}{.12,.54,.11}

    % ANSI colors
    \definecolor{ansi-black}{HTML}{3E424D}
    \definecolor{ansi-black-intense}{HTML}{282C36}
    \definecolor{ansi-red}{HTML}{E75C58}
    \definecolor{ansi-red-intense}{HTML}{B22B31}
    \definecolor{ansi-green}{HTML}{00A250}
    \definecolor{ansi-green-intense}{HTML}{007427}
    \definecolor{ansi-yellow}{HTML}{DDB62B}
    \definecolor{ansi-yellow-intense}{HTML}{B27D12}
    \definecolor{ansi-blue}{HTML}{208FFB}
    \definecolor{ansi-blue-intense}{HTML}{0065CA}
    \definecolor{ansi-magenta}{HTML}{D160C4}
    \definecolor{ansi-magenta-intense}{HTML}{A03196}
    \definecolor{ansi-cyan}{HTML}{60C6C8}
    \definecolor{ansi-cyan-intense}{HTML}{258F8F}
    \definecolor{ansi-white}{HTML}{C5C1B4}
    \definecolor{ansi-white-intense}{HTML}{A1A6B2}
    \definecolor{ansi-default-inverse-fg}{HTML}{FFFFFF}
    \definecolor{ansi-default-inverse-bg}{HTML}{000000}

    % commands and environments needed by pandoc snippets
    % extracted from the output of `pandoc -s`
    \providecommand{\tightlist}{%
      \setlength{\itemsep}{0pt}\setlength{\parskip}{0pt}}
    \DefineVerbatimEnvironment{Highlighting}{Verbatim}{commandchars=\\\{\}}
    % Add ',fontsize=\small' for more characters per line
    \newenvironment{Shaded}{}{}
    \newcommand{\KeywordTok}[1]{\textcolor[rgb]{0.00,0.44,0.13}{\textbf{{#1}}}}
    \newcommand{\DataTypeTok}[1]{\textcolor[rgb]{0.56,0.13,0.00}{{#1}}}
    \newcommand{\DecValTok}[1]{\textcolor[rgb]{0.25,0.63,0.44}{{#1}}}
    \newcommand{\BaseNTok}[1]{\textcolor[rgb]{0.25,0.63,0.44}{{#1}}}
    \newcommand{\FloatTok}[1]{\textcolor[rgb]{0.25,0.63,0.44}{{#1}}}
    \newcommand{\CharTok}[1]{\textcolor[rgb]{0.25,0.44,0.63}{{#1}}}
    \newcommand{\StringTok}[1]{\textcolor[rgb]{0.25,0.44,0.63}{{#1}}}
    \newcommand{\CommentTok}[1]{\textcolor[rgb]{0.38,0.63,0.69}{\textit{{#1}}}}
    \newcommand{\OtherTok}[1]{\textcolor[rgb]{0.00,0.44,0.13}{{#1}}}
    \newcommand{\AlertTok}[1]{\textcolor[rgb]{1.00,0.00,0.00}{\textbf{{#1}}}}
    \newcommand{\FunctionTok}[1]{\textcolor[rgb]{0.02,0.16,0.49}{{#1}}}
    \newcommand{\RegionMarkerTok}[1]{{#1}}
    \newcommand{\ErrorTok}[1]{\textcolor[rgb]{1.00,0.00,0.00}{\textbf{{#1}}}}
    \newcommand{\NormalTok}[1]{{#1}}
    
    % Additional commands for more recent versions of Pandoc
    \newcommand{\ConstantTok}[1]{\textcolor[rgb]{0.53,0.00,0.00}{{#1}}}
    \newcommand{\SpecialCharTok}[1]{\textcolor[rgb]{0.25,0.44,0.63}{{#1}}}
    \newcommand{\VerbatimStringTok}[1]{\textcolor[rgb]{0.25,0.44,0.63}{{#1}}}
    \newcommand{\SpecialStringTok}[1]{\textcolor[rgb]{0.73,0.40,0.53}{{#1}}}
    \newcommand{\ImportTok}[1]{{#1}}
    \newcommand{\DocumentationTok}[1]{\textcolor[rgb]{0.73,0.13,0.13}{\textit{{#1}}}}
    \newcommand{\AnnotationTok}[1]{\textcolor[rgb]{0.38,0.63,0.69}{\textbf{\textit{{#1}}}}}
    \newcommand{\CommentVarTok}[1]{\textcolor[rgb]{0.38,0.63,0.69}{\textbf{\textit{{#1}}}}}
    \newcommand{\VariableTok}[1]{\textcolor[rgb]{0.10,0.09,0.49}{{#1}}}
    \newcommand{\ControlFlowTok}[1]{\textcolor[rgb]{0.00,0.44,0.13}{\textbf{{#1}}}}
    \newcommand{\OperatorTok}[1]{\textcolor[rgb]{0.40,0.40,0.40}{{#1}}}
    \newcommand{\BuiltInTok}[1]{{#1}}
    \newcommand{\ExtensionTok}[1]{{#1}}
    \newcommand{\PreprocessorTok}[1]{\textcolor[rgb]{0.74,0.48,0.00}{{#1}}}
    \newcommand{\AttributeTok}[1]{\textcolor[rgb]{0.49,0.56,0.16}{{#1}}}
    \newcommand{\InformationTok}[1]{\textcolor[rgb]{0.38,0.63,0.69}{\textbf{\textit{{#1}}}}}
    \newcommand{\WarningTok}[1]{\textcolor[rgb]{0.38,0.63,0.69}{\textbf{\textit{{#1}}}}}
    
    
    % Define a nice break command that doesn't care if a line doesn't already
    % exist.
    \def\br{\hspace*{\fill} \\* }
    % Math Jax compatibility definitions
    \def\gt{>}
    \def\lt{<}
    \let\Oldtex\TeX
    \let\Oldlatex\LaTeX
    \renewcommand{\TeX}{\textrm{\Oldtex}}
    \renewcommand{\LaTeX}{\textrm{\Oldlatex}}
    % Document parameters
    % Document title
    \title{DB0201EN-Week4-2-2-PeerAssign-v5-py}
    
    
    
    
    
% Pygments definitions
\makeatletter
\def\PY@reset{\let\PY@it=\relax \let\PY@bf=\relax%
    \let\PY@ul=\relax \let\PY@tc=\relax%
    \let\PY@bc=\relax \let\PY@ff=\relax}
\def\PY@tok#1{\csname PY@tok@#1\endcsname}
\def\PY@toks#1+{\ifx\relax#1\empty\else%
    \PY@tok{#1}\expandafter\PY@toks\fi}
\def\PY@do#1{\PY@bc{\PY@tc{\PY@ul{%
    \PY@it{\PY@bf{\PY@ff{#1}}}}}}}
\def\PY#1#2{\PY@reset\PY@toks#1+\relax+\PY@do{#2}}

\expandafter\def\csname PY@tok@w\endcsname{\def\PY@tc##1{\textcolor[rgb]{0.73,0.73,0.73}{##1}}}
\expandafter\def\csname PY@tok@c\endcsname{\let\PY@it=\textit\def\PY@tc##1{\textcolor[rgb]{0.25,0.50,0.50}{##1}}}
\expandafter\def\csname PY@tok@cp\endcsname{\def\PY@tc##1{\textcolor[rgb]{0.74,0.48,0.00}{##1}}}
\expandafter\def\csname PY@tok@k\endcsname{\let\PY@bf=\textbf\def\PY@tc##1{\textcolor[rgb]{0.00,0.50,0.00}{##1}}}
\expandafter\def\csname PY@tok@kp\endcsname{\def\PY@tc##1{\textcolor[rgb]{0.00,0.50,0.00}{##1}}}
\expandafter\def\csname PY@tok@kt\endcsname{\def\PY@tc##1{\textcolor[rgb]{0.69,0.00,0.25}{##1}}}
\expandafter\def\csname PY@tok@o\endcsname{\def\PY@tc##1{\textcolor[rgb]{0.40,0.40,0.40}{##1}}}
\expandafter\def\csname PY@tok@ow\endcsname{\let\PY@bf=\textbf\def\PY@tc##1{\textcolor[rgb]{0.67,0.13,1.00}{##1}}}
\expandafter\def\csname PY@tok@nb\endcsname{\def\PY@tc##1{\textcolor[rgb]{0.00,0.50,0.00}{##1}}}
\expandafter\def\csname PY@tok@nf\endcsname{\def\PY@tc##1{\textcolor[rgb]{0.00,0.00,1.00}{##1}}}
\expandafter\def\csname PY@tok@nc\endcsname{\let\PY@bf=\textbf\def\PY@tc##1{\textcolor[rgb]{0.00,0.00,1.00}{##1}}}
\expandafter\def\csname PY@tok@nn\endcsname{\let\PY@bf=\textbf\def\PY@tc##1{\textcolor[rgb]{0.00,0.00,1.00}{##1}}}
\expandafter\def\csname PY@tok@ne\endcsname{\let\PY@bf=\textbf\def\PY@tc##1{\textcolor[rgb]{0.82,0.25,0.23}{##1}}}
\expandafter\def\csname PY@tok@nv\endcsname{\def\PY@tc##1{\textcolor[rgb]{0.10,0.09,0.49}{##1}}}
\expandafter\def\csname PY@tok@no\endcsname{\def\PY@tc##1{\textcolor[rgb]{0.53,0.00,0.00}{##1}}}
\expandafter\def\csname PY@tok@nl\endcsname{\def\PY@tc##1{\textcolor[rgb]{0.63,0.63,0.00}{##1}}}
\expandafter\def\csname PY@tok@ni\endcsname{\let\PY@bf=\textbf\def\PY@tc##1{\textcolor[rgb]{0.60,0.60,0.60}{##1}}}
\expandafter\def\csname PY@tok@na\endcsname{\def\PY@tc##1{\textcolor[rgb]{0.49,0.56,0.16}{##1}}}
\expandafter\def\csname PY@tok@nt\endcsname{\let\PY@bf=\textbf\def\PY@tc##1{\textcolor[rgb]{0.00,0.50,0.00}{##1}}}
\expandafter\def\csname PY@tok@nd\endcsname{\def\PY@tc##1{\textcolor[rgb]{0.67,0.13,1.00}{##1}}}
\expandafter\def\csname PY@tok@s\endcsname{\def\PY@tc##1{\textcolor[rgb]{0.73,0.13,0.13}{##1}}}
\expandafter\def\csname PY@tok@sd\endcsname{\let\PY@it=\textit\def\PY@tc##1{\textcolor[rgb]{0.73,0.13,0.13}{##1}}}
\expandafter\def\csname PY@tok@si\endcsname{\let\PY@bf=\textbf\def\PY@tc##1{\textcolor[rgb]{0.73,0.40,0.53}{##1}}}
\expandafter\def\csname PY@tok@se\endcsname{\let\PY@bf=\textbf\def\PY@tc##1{\textcolor[rgb]{0.73,0.40,0.13}{##1}}}
\expandafter\def\csname PY@tok@sr\endcsname{\def\PY@tc##1{\textcolor[rgb]{0.73,0.40,0.53}{##1}}}
\expandafter\def\csname PY@tok@ss\endcsname{\def\PY@tc##1{\textcolor[rgb]{0.10,0.09,0.49}{##1}}}
\expandafter\def\csname PY@tok@sx\endcsname{\def\PY@tc##1{\textcolor[rgb]{0.00,0.50,0.00}{##1}}}
\expandafter\def\csname PY@tok@m\endcsname{\def\PY@tc##1{\textcolor[rgb]{0.40,0.40,0.40}{##1}}}
\expandafter\def\csname PY@tok@gh\endcsname{\let\PY@bf=\textbf\def\PY@tc##1{\textcolor[rgb]{0.00,0.00,0.50}{##1}}}
\expandafter\def\csname PY@tok@gu\endcsname{\let\PY@bf=\textbf\def\PY@tc##1{\textcolor[rgb]{0.50,0.00,0.50}{##1}}}
\expandafter\def\csname PY@tok@gd\endcsname{\def\PY@tc##1{\textcolor[rgb]{0.63,0.00,0.00}{##1}}}
\expandafter\def\csname PY@tok@gi\endcsname{\def\PY@tc##1{\textcolor[rgb]{0.00,0.63,0.00}{##1}}}
\expandafter\def\csname PY@tok@gr\endcsname{\def\PY@tc##1{\textcolor[rgb]{1.00,0.00,0.00}{##1}}}
\expandafter\def\csname PY@tok@ge\endcsname{\let\PY@it=\textit}
\expandafter\def\csname PY@tok@gs\endcsname{\let\PY@bf=\textbf}
\expandafter\def\csname PY@tok@gp\endcsname{\let\PY@bf=\textbf\def\PY@tc##1{\textcolor[rgb]{0.00,0.00,0.50}{##1}}}
\expandafter\def\csname PY@tok@go\endcsname{\def\PY@tc##1{\textcolor[rgb]{0.53,0.53,0.53}{##1}}}
\expandafter\def\csname PY@tok@gt\endcsname{\def\PY@tc##1{\textcolor[rgb]{0.00,0.27,0.87}{##1}}}
\expandafter\def\csname PY@tok@err\endcsname{\def\PY@bc##1{\setlength{\fboxsep}{0pt}\fcolorbox[rgb]{1.00,0.00,0.00}{1,1,1}{\strut ##1}}}
\expandafter\def\csname PY@tok@kc\endcsname{\let\PY@bf=\textbf\def\PY@tc##1{\textcolor[rgb]{0.00,0.50,0.00}{##1}}}
\expandafter\def\csname PY@tok@kd\endcsname{\let\PY@bf=\textbf\def\PY@tc##1{\textcolor[rgb]{0.00,0.50,0.00}{##1}}}
\expandafter\def\csname PY@tok@kn\endcsname{\let\PY@bf=\textbf\def\PY@tc##1{\textcolor[rgb]{0.00,0.50,0.00}{##1}}}
\expandafter\def\csname PY@tok@kr\endcsname{\let\PY@bf=\textbf\def\PY@tc##1{\textcolor[rgb]{0.00,0.50,0.00}{##1}}}
\expandafter\def\csname PY@tok@bp\endcsname{\def\PY@tc##1{\textcolor[rgb]{0.00,0.50,0.00}{##1}}}
\expandafter\def\csname PY@tok@fm\endcsname{\def\PY@tc##1{\textcolor[rgb]{0.00,0.00,1.00}{##1}}}
\expandafter\def\csname PY@tok@vc\endcsname{\def\PY@tc##1{\textcolor[rgb]{0.10,0.09,0.49}{##1}}}
\expandafter\def\csname PY@tok@vg\endcsname{\def\PY@tc##1{\textcolor[rgb]{0.10,0.09,0.49}{##1}}}
\expandafter\def\csname PY@tok@vi\endcsname{\def\PY@tc##1{\textcolor[rgb]{0.10,0.09,0.49}{##1}}}
\expandafter\def\csname PY@tok@vm\endcsname{\def\PY@tc##1{\textcolor[rgb]{0.10,0.09,0.49}{##1}}}
\expandafter\def\csname PY@tok@sa\endcsname{\def\PY@tc##1{\textcolor[rgb]{0.73,0.13,0.13}{##1}}}
\expandafter\def\csname PY@tok@sb\endcsname{\def\PY@tc##1{\textcolor[rgb]{0.73,0.13,0.13}{##1}}}
\expandafter\def\csname PY@tok@sc\endcsname{\def\PY@tc##1{\textcolor[rgb]{0.73,0.13,0.13}{##1}}}
\expandafter\def\csname PY@tok@dl\endcsname{\def\PY@tc##1{\textcolor[rgb]{0.73,0.13,0.13}{##1}}}
\expandafter\def\csname PY@tok@s2\endcsname{\def\PY@tc##1{\textcolor[rgb]{0.73,0.13,0.13}{##1}}}
\expandafter\def\csname PY@tok@sh\endcsname{\def\PY@tc##1{\textcolor[rgb]{0.73,0.13,0.13}{##1}}}
\expandafter\def\csname PY@tok@s1\endcsname{\def\PY@tc##1{\textcolor[rgb]{0.73,0.13,0.13}{##1}}}
\expandafter\def\csname PY@tok@mb\endcsname{\def\PY@tc##1{\textcolor[rgb]{0.40,0.40,0.40}{##1}}}
\expandafter\def\csname PY@tok@mf\endcsname{\def\PY@tc##1{\textcolor[rgb]{0.40,0.40,0.40}{##1}}}
\expandafter\def\csname PY@tok@mh\endcsname{\def\PY@tc##1{\textcolor[rgb]{0.40,0.40,0.40}{##1}}}
\expandafter\def\csname PY@tok@mi\endcsname{\def\PY@tc##1{\textcolor[rgb]{0.40,0.40,0.40}{##1}}}
\expandafter\def\csname PY@tok@il\endcsname{\def\PY@tc##1{\textcolor[rgb]{0.40,0.40,0.40}{##1}}}
\expandafter\def\csname PY@tok@mo\endcsname{\def\PY@tc##1{\textcolor[rgb]{0.40,0.40,0.40}{##1}}}
\expandafter\def\csname PY@tok@ch\endcsname{\let\PY@it=\textit\def\PY@tc##1{\textcolor[rgb]{0.25,0.50,0.50}{##1}}}
\expandafter\def\csname PY@tok@cm\endcsname{\let\PY@it=\textit\def\PY@tc##1{\textcolor[rgb]{0.25,0.50,0.50}{##1}}}
\expandafter\def\csname PY@tok@cpf\endcsname{\let\PY@it=\textit\def\PY@tc##1{\textcolor[rgb]{0.25,0.50,0.50}{##1}}}
\expandafter\def\csname PY@tok@c1\endcsname{\let\PY@it=\textit\def\PY@tc##1{\textcolor[rgb]{0.25,0.50,0.50}{##1}}}
\expandafter\def\csname PY@tok@cs\endcsname{\let\PY@it=\textit\def\PY@tc##1{\textcolor[rgb]{0.25,0.50,0.50}{##1}}}

\def\PYZbs{\char`\\}
\def\PYZus{\char`\_}
\def\PYZob{\char`\{}
\def\PYZcb{\char`\}}
\def\PYZca{\char`\^}
\def\PYZam{\char`\&}
\def\PYZlt{\char`\<}
\def\PYZgt{\char`\>}
\def\PYZsh{\char`\#}
\def\PYZpc{\char`\%}
\def\PYZdl{\char`\$}
\def\PYZhy{\char`\-}
\def\PYZsq{\char`\'}
\def\PYZdq{\char`\"}
\def\PYZti{\char`\~}
% for compatibility with earlier versions
\def\PYZat{@}
\def\PYZlb{[}
\def\PYZrb{]}
\makeatother


    % For linebreaks inside Verbatim environment from package fancyvrb. 
    \makeatletter
        \newbox\Wrappedcontinuationbox 
        \newbox\Wrappedvisiblespacebox 
        \newcommand*\Wrappedvisiblespace {\textcolor{red}{\textvisiblespace}} 
        \newcommand*\Wrappedcontinuationsymbol {\textcolor{red}{\llap{\tiny$\m@th\hookrightarrow$}}} 
        \newcommand*\Wrappedcontinuationindent {3ex } 
        \newcommand*\Wrappedafterbreak {\kern\Wrappedcontinuationindent\copy\Wrappedcontinuationbox} 
        % Take advantage of the already applied Pygments mark-up to insert 
        % potential linebreaks for TeX processing. 
        %        {, <, #, %, $, ' and ": go to next line. 
        %        _, }, ^, &, >, - and ~: stay at end of broken line. 
        % Use of \textquotesingle for straight quote. 
        \newcommand*\Wrappedbreaksatspecials {% 
            \def\PYGZus{\discretionary{\char`\_}{\Wrappedafterbreak}{\char`\_}}% 
            \def\PYGZob{\discretionary{}{\Wrappedafterbreak\char`\{}{\char`\{}}% 
            \def\PYGZcb{\discretionary{\char`\}}{\Wrappedafterbreak}{\char`\}}}% 
            \def\PYGZca{\discretionary{\char`\^}{\Wrappedafterbreak}{\char`\^}}% 
            \def\PYGZam{\discretionary{\char`\&}{\Wrappedafterbreak}{\char`\&}}% 
            \def\PYGZlt{\discretionary{}{\Wrappedafterbreak\char`\<}{\char`\<}}% 
            \def\PYGZgt{\discretionary{\char`\>}{\Wrappedafterbreak}{\char`\>}}% 
            \def\PYGZsh{\discretionary{}{\Wrappedafterbreak\char`\#}{\char`\#}}% 
            \def\PYGZpc{\discretionary{}{\Wrappedafterbreak\char`\%}{\char`\%}}% 
            \def\PYGZdl{\discretionary{}{\Wrappedafterbreak\char`\$}{\char`\$}}% 
            \def\PYGZhy{\discretionary{\char`\-}{\Wrappedafterbreak}{\char`\-}}% 
            \def\PYGZsq{\discretionary{}{\Wrappedafterbreak\textquotesingle}{\textquotesingle}}% 
            \def\PYGZdq{\discretionary{}{\Wrappedafterbreak\char`\"}{\char`\"}}% 
            \def\PYGZti{\discretionary{\char`\~}{\Wrappedafterbreak}{\char`\~}}% 
        } 
        % Some characters . , ; ? ! / are not pygmentized. 
        % This macro makes them "active" and they will insert potential linebreaks 
        \newcommand*\Wrappedbreaksatpunct {% 
            \lccode`\~`\.\lowercase{\def~}{\discretionary{\hbox{\char`\.}}{\Wrappedafterbreak}{\hbox{\char`\.}}}% 
            \lccode`\~`\,\lowercase{\def~}{\discretionary{\hbox{\char`\,}}{\Wrappedafterbreak}{\hbox{\char`\,}}}% 
            \lccode`\~`\;\lowercase{\def~}{\discretionary{\hbox{\char`\;}}{\Wrappedafterbreak}{\hbox{\char`\;}}}% 
            \lccode`\~`\:\lowercase{\def~}{\discretionary{\hbox{\char`\:}}{\Wrappedafterbreak}{\hbox{\char`\:}}}% 
            \lccode`\~`\?\lowercase{\def~}{\discretionary{\hbox{\char`\?}}{\Wrappedafterbreak}{\hbox{\char`\?}}}% 
            \lccode`\~`\!\lowercase{\def~}{\discretionary{\hbox{\char`\!}}{\Wrappedafterbreak}{\hbox{\char`\!}}}% 
            \lccode`\~`\/\lowercase{\def~}{\discretionary{\hbox{\char`\/}}{\Wrappedafterbreak}{\hbox{\char`\/}}}% 
            \catcode`\.\active
            \catcode`\,\active 
            \catcode`\;\active
            \catcode`\:\active
            \catcode`\?\active
            \catcode`\!\active
            \catcode`\/\active 
            \lccode`\~`\~ 	
        }
    \makeatother

    \let\OriginalVerbatim=\Verbatim
    \makeatletter
    \renewcommand{\Verbatim}[1][1]{%
        %\parskip\z@skip
        \sbox\Wrappedcontinuationbox {\Wrappedcontinuationsymbol}%
        \sbox\Wrappedvisiblespacebox {\FV@SetupFont\Wrappedvisiblespace}%
        \def\FancyVerbFormatLine ##1{\hsize\linewidth
            \vtop{\raggedright\hyphenpenalty\z@\exhyphenpenalty\z@
                \doublehyphendemerits\z@\finalhyphendemerits\z@
                \strut ##1\strut}%
        }%
        % If the linebreak is at a space, the latter will be displayed as visible
        % space at end of first line, and a continuation symbol starts next line.
        % Stretch/shrink are however usually zero for typewriter font.
        \def\FV@Space {%
            \nobreak\hskip\z@ plus\fontdimen3\font minus\fontdimen4\font
            \discretionary{\copy\Wrappedvisiblespacebox}{\Wrappedafterbreak}
            {\kern\fontdimen2\font}%
        }%
        
        % Allow breaks at special characters using \PYG... macros.
        \Wrappedbreaksatspecials
        % Breaks at punctuation characters . , ; ? ! and / need catcode=\active 	
        \OriginalVerbatim[#1,codes*=\Wrappedbreaksatpunct]%
    }
    \makeatother

    % Exact colors from NB
    \definecolor{incolor}{HTML}{303F9F}
    \definecolor{outcolor}{HTML}{D84315}
    \definecolor{cellborder}{HTML}{CFCFCF}
    \definecolor{cellbackground}{HTML}{F7F7F7}
    
    % prompt
    \makeatletter
    \newcommand{\boxspacing}{\kern\kvtcb@left@rule\kern\kvtcb@boxsep}
    \makeatother
    \newcommand{\prompt}[4]{
        \ttfamily\llap{{\color{#2}[#3]:\hspace{3pt}#4}}\vspace{-\baselineskip}
    }
    

    
    % Prevent overflowing lines due to hard-to-break entities
    \sloppy 
    % Setup hyperref package
    \hypersetup{
      breaklinks=true,  % so long urls are correctly broken across lines
      colorlinks=true,
      urlcolor=urlcolor,
      linkcolor=linkcolor,
      citecolor=citecolor,
      }
    % Slightly bigger margins than the latex defaults
    
    \geometry{verbose,tmargin=1in,bmargin=1in,lmargin=1in,rmargin=1in}
    
    

\begin{document}
    
    \maketitle
    
    

    
    Assignment: Notebook for Peer Assignment

    \hypertarget{introduction}{%
\section{Introduction}\label{introduction}}

Using this Python notebook you will: 1. Understand 3 Chicago datasets\\
1. Load the 3 datasets into 3 tables in a Db2 database 1. Execute SQL
queries to answer assignment questions

    \hypertarget{understand-the-datasets}{%
\subsection{Understand the datasets}\label{understand-the-datasets}}

To complete the assignment problems in this notebook you will be using
three datasets that are available on the city of Chicago's Data Portal:
1. Socioeconomic Indicators in Chicago 1. Chicago Public Schools 1.
Chicago Crime Data

\hypertarget{socioeconomic-indicators-in-chicago}{%
\subsubsection{1. Socioeconomic Indicators in
Chicago}\label{socioeconomic-indicators-in-chicago}}

This dataset contains a selection of six socioeconomic indicators of
public health significance and a ``hardship index,'' for each Chicago
community area, for the years 2008 -- 2012.

For this assignment you will use a snapshot of this dataset which can be
downloaded from:
https://ibm.box.com/shared/static/05c3415cbfbtfnr2fx4atenb2sd361ze.csv

A detailed description of this dataset and the original dataset can be
obtained from the Chicago Data Portal at:
https://data.cityofchicago.org/Health-Human-Services/Census-Data-Selected-socioeconomic-indicators-in-C/kn9c-c2s2

\hypertarget{chicago-public-schools}{%
\subsubsection{2. Chicago Public Schools}\label{chicago-public-schools}}

This dataset shows all school level performance data used to create CPS
School Report Cards for the 2011-2012 school year. This dataset is
provided by the city of Chicago's Data Portal.

For this assignment you will use a snapshot of this dataset which can be
downloaded from:
https://ibm.box.com/shared/static/f9gjvj1gjmxxzycdhplzt01qtz0s7ew7.csv

A detailed description of this dataset and the original dataset can be
obtained from the Chicago Data Portal at:
https://data.cityofchicago.org/Education/Chicago-Public-Schools-Progress-Report-Cards-2011-/9xs2-f89t

\hypertarget{chicago-crime-data}{%
\subsubsection{3. Chicago Crime Data}\label{chicago-crime-data}}

This dataset reflects reported incidents of crime (with the exception of
murders where data exists for each victim) that occurred in the City of
Chicago from 2001 to present, minus the most recent seven days.

This dataset is quite large - over 1.5GB in size with over 6.5 million
rows. For the purposes of this assignment we will use a much smaller
sample of this dataset which can be downloaded from:
https://ibm.box.com/shared/static/svflyugsr9zbqy5bmowgswqemfpm1x7f.csv

A detailed description of this dataset and the original dataset can be
obtained from the Chicago Data Portal at:
https://data.cityofchicago.org/Public-Safety/Crimes-2001-to-present/ijzp-q8t2

    \hypertarget{download-the-datasets}{%
\subsubsection{Download the datasets}\label{download-the-datasets}}

In many cases the dataset to be analyzed is available as a .CSV (comma
separated values) file, perhaps on the internet. Click on the links
below to download and save the datasets (.CSV files): 1.
\textbf{CENSUS\_DATA:}
https://ibm.box.com/shared/static/05c3415cbfbtfnr2fx4atenb2sd361ze.csv
1. \textbf{CHICAGO\_PUBLIC\_SCHOOLS}
https://ibm.box.com/shared/static/f9gjvj1gjmxxzycdhplzt01qtz0s7ew7.csv
1. \textbf{CHICAGO\_CRIME\_DATA:}
https://ibm.box.com/shared/static/svflyugsr9zbqy5bmowgswqemfpm1x7f.csv

\textbf{NOTE:} Ensure you have downloaded the datasets using the links
above instead of directly from the Chicago Data Portal. The versions
linked here are subsets of the original datasets and have some of the
column names modified to be more database friendly which will make it
easier to complete this assignment.

    \hypertarget{store-the-datasets-in-database-tables}{%
\subsubsection{Store the datasets in database
tables}\label{store-the-datasets-in-database-tables}}

To analyze the data using SQL, it first needs to be stored in the
database.

While it is easier to read the dataset into a Pandas dataframe and then
PERSIST it into the database as we saw in Week 3 Lab 3, it results in
mapping to default datatypes which may not be optimal for SQL querying.
For example a long textual field may map to a CLOB instead of a VARCHAR.

Therefore, \textbf{it is highly recommended to manually load the table
using the database console LOAD tool, as indicated in Week 2 Lab 1 Part
II}. The only difference with that lab is that in Step 5 of the
instructions you will need to click on create ``(+) New Table'' and
specify the name of the table you want to create and then click
``Next''.

\hypertarget{now-open-the-db2-console-open-the-load-tool-select-drag-the-.csv-file-for-the-first-dataset-next-create-a-new-table-and-then-follow-the-steps-on-screen-instructions-to-load-the-data.-name-the-new-tables-as-folows}{%
\subparagraph{Now open the Db2 console, open the LOAD tool, Select /
Drag the .CSV file for the first dataset, Next create a New Table, and
then follow the steps on-screen instructions to load the data. Name the
new tables as
folows:}\label{now-open-the-db2-console-open-the-load-tool-select-drag-the-.csv-file-for-the-first-dataset-next-create-a-new-table-and-then-follow-the-steps-on-screen-instructions-to-load-the-data.-name-the-new-tables-as-folows}}

\begin{enumerate}
\def\labelenumi{\arabic{enumi}.}
\tightlist
\item
  \textbf{CENSUS\_DATA}
\item
  \textbf{CHICAGO\_PUBLIC\_SCHOOLS}
\item
  \textbf{CHICAGO\_CRIME\_DATA}
\end{enumerate}

    \hypertarget{connect-to-the-database}{%
\subsubsection{Connect to the database}\label{connect-to-the-database}}

Let us first load the SQL extension and establish a connection with the
database

    \begin{tcolorbox}[breakable, size=fbox, boxrule=1pt, pad at break*=1mm,colback=cellbackground, colframe=cellborder]
\prompt{In}{incolor}{2}{\boxspacing}
\begin{Verbatim}[commandchars=\\\{\}]
\PY{o}{\PYZpc{}}\PY{k}{load\PYZus{}ext} sql
\end{Verbatim}
\end{tcolorbox}

    In the next cell enter your db2 connection string. Recall you created
Service Credentials for your Db2 instance in first lab in Week 3. From
the \textbf{uri} field of your Db2 service credentials copy everything
after db2:// (except the double quote at the end) and paste it in the
cell below after ibm\_db\_sa://

    \begin{tcolorbox}[breakable, size=fbox, boxrule=1pt, pad at break*=1mm,colback=cellbackground, colframe=cellborder]
\prompt{In}{incolor}{3}{\boxspacing}
\begin{Verbatim}[commandchars=\\\{\}]
\PY{c+c1}{\PYZsh{} Remember the connection string is of the format:}
\PY{c+c1}{\PYZsh{} \PYZpc{}sql ibm\PYZus{}db\PYZus{}sa://my\PYZhy{}username:my\PYZhy{}password@my\PYZhy{}hostname:my\PYZhy{}port/my\PYZhy{}db\PYZhy{}name}
\PY{c+c1}{\PYZsh{} Enter the connection string for your Db2 on Cloud database instance below}
\PY{o}{\PYZpc{}}\PY{k}{sql} ibm\PYZus{}db\PYZus{}sa://my\PYZhy{}username:my\PYZhy{}password@my\PYZhy{}hostname:my\PYZhy{}port/my\PYZhy{}db\PYZhy{}name
\end{Verbatim}
\end{tcolorbox}

            \begin{tcolorbox}[breakable, size=fbox, boxrule=.5pt, pad at break*=1mm, opacityfill=0]
\prompt{Out}{outcolor}{3}{\boxspacing}
\begin{Verbatim}[commandchars=\\\{\}]
'Connected: nrx71347@BLUDB'
\end{Verbatim}
\end{tcolorbox}
        
    \hypertarget{problems}{%
\subsection{Problems}\label{problems}}

Now write and execute SQL queries to solve assignment problems

\hypertarget{problem-1}{%
\subsubsection{Problem 1}\label{problem-1}}

\hypertarget{find-the-total-number-of-crimes-recorded-in-the-crime-table}{%
\subparagraph{Find the total number of crimes recorded in the CRIME
table}\label{find-the-total-number-of-crimes-recorded-in-the-crime-table}}

    \begin{tcolorbox}[breakable, size=fbox, boxrule=1pt, pad at break*=1mm,colback=cellbackground, colframe=cellborder]
\prompt{In}{incolor}{9}{\boxspacing}
\begin{Verbatim}[commandchars=\\\{\}]
\PY{c+c1}{\PYZsh{} Rows in Crime table}
\PY{o}{\PYZpc{}}\PY{k}{sql} select count(*) as total\PYZus{}crimes from CRIME\PYZus{}DATA
\end{Verbatim}
\end{tcolorbox}

    \begin{Verbatim}[commandchars=\\\{\}]
 * ibm\_db\_sa://nrx71347:***@dashdb-txn-sbox-yp-lon02-07.services.eu-
gb.bluemix.net:50000/BLUDB
Done.
    \end{Verbatim}

            \begin{tcolorbox}[breakable, size=fbox, boxrule=.5pt, pad at break*=1mm, opacityfill=0]
\prompt{Out}{outcolor}{9}{\boxspacing}
\begin{Verbatim}[commandchars=\\\{\}]
[(Decimal('533'),)]
\end{Verbatim}
\end{tcolorbox}
        
    \hypertarget{problem-2}{%
\subsubsection{Problem 2}\label{problem-2}}

\hypertarget{retrieve-first-10-rows-from-the-crime-table}{%
\subparagraph{Retrieve first 10 rows from the CRIME
table}\label{retrieve-first-10-rows-from-the-crime-table}}

    \begin{tcolorbox}[breakable, size=fbox, boxrule=1pt, pad at break*=1mm,colback=cellbackground, colframe=cellborder]
\prompt{In}{incolor}{6}{\boxspacing}
\begin{Verbatim}[commandchars=\\\{\}]
\PY{o}{\PYZpc{}}\PY{k}{sql} SELECT * FROM CRIME\PYZus{}DATA LIMIT 10
\end{Verbatim}
\end{tcolorbox}

    \begin{Verbatim}[commandchars=\\\{\}]
 * ibm\_db\_sa://nrx71347:***@dashdb-txn-sbox-yp-lon02-07.services.eu-
gb.bluemix.net:50000/BLUDB
Done.
    \end{Verbatim}

            \begin{tcolorbox}[breakable, size=fbox, boxrule=.5pt, pad at break*=1mm, opacityfill=0]
\prompt{Out}{outcolor}{6}{\boxspacing}
\begin{Verbatim}[commandchars=\\\{\}]
[(3512276, 'HK587712', datetime.datetime(2004, 8, 28, 17, 50, 56), '047XX S
KEDZIE AVE', '890', 'THEFT', 'FROM BUILDING', 'SMALL RETAIL STORE', 'FALSE',
'FALSE', 911, 9, 14, 58, '6', 1155838, 1873050, 2004, datetime.datetime(2018, 2,
10, 15, 50, 1), Decimal('41.80744050'), Decimal('-87.70395585'), '(41.8074405,
-87.703955849)'),
 (3406613, 'HK456306', datetime.datetime(2004, 6, 26, 12, 40), '009XX N CENTRAL
PARK AVE', '820', 'THEFT', '\$500 AND UNDER', 'OTHER', 'FALSE', 'FALSE', 1112,
11, 27, 23, '6', 1152206, 1906127, 2004, datetime.datetime(2018, 2, 28, 15, 56,
25), Decimal('41.89827996'), Decimal('-87.71640551'), '(41.898279962,
-87.716405505)'),
 (8002131, 'HT233595', datetime.datetime(2011, 4, 4, 5, 45), '043XX S WABASH
AVE', '820', 'THEFT', '\$500 AND UNDER', 'NURSING HOME/RETIREMENT HOME', 'FALSE',
'FALSE', 221, 2, 3, 38, '6', 1177436, 1876313, 2011, datetime.datetime(2018, 2,
10, 15, 50, 1), Decimal('41.81593313'), Decimal('-87.62464213'), '(41.815933131,
-87.624642127)'),
 (7903289, 'HT133522', datetime.datetime(2010, 12, 30, 16, 30), '083XX S
KINGSTON AVE', '840', 'THEFT', 'FINANCIAL ID THEFT: OVER \$300', 'RESIDENCE',
'FALSE', 'FALSE', 423, 4, 7, 46, '6', 1194622, 1850125, 2010,
datetime.datetime(2018, 2, 10, 15, 50, 1), Decimal('41.74366532'),
Decimal('-87.56246276'), '(41.743665322, -87.562462756)'),
 (10402076, 'HZ138551', datetime.datetime(2016, 2, 2, 19, 30), '033XX W 66TH
ST', '820', 'THEFT', '\$500 AND UNDER', 'ALLEY', 'FALSE', 'FALSE', 831, 8, 15,
66, '6', 1155240, 1860661, 2016, datetime.datetime(2018, 2, 10, 15, 50, 1),
Decimal('41.77345530'), Decimal('-87.70648047'), '(41.773455295,
-87.706480471)'),
 (7732712, 'HS540106', datetime.datetime(2010, 9, 29, 7, 59), '006XX W CHICAGO
AVE', '810', 'THEFT', 'OVER \$500', 'PARKING LOT/GARAGE(NON.RESID.)', 'FALSE',
'FALSE', 1323, 12, 27, 24, '6', 1171668, 1905607, 2010, datetime.datetime(2018,
2, 10, 15, 50, 1), Decimal('41.89644677'), Decimal('-87.64493868'),
'(41.896446772, -87.644938678)'),
 (10769475, 'HZ534771', datetime.datetime(2016, 11, 30, 1, 15), '050XX N KEDZIE
AVE', '810', 'THEFT', 'OVER \$500', 'STREET', 'FALSE', 'FALSE', 1713, 17, 33, 14,
'6', 1154133, 1933314, 2016, datetime.datetime(2018, 2, 10, 15, 50, 1),
Decimal('41.97284491'), Decimal('-87.70860008'), '(41.972844913,
-87.708600079)'),
 (4494340, 'HL793243', datetime.datetime(2005, 12, 16, 16, 45), '005XX E
PERSHING RD', '860', 'THEFT', 'RETAIL THEFT', 'GROCERY FOOD STORE', 'TRUE',
'FALSE', 213, 2, 3, 38, '6', 1180448, 1879234, 2005, datetime.datetime(2018, 2,
28, 15, 56, 25), Decimal('41.82387989'), Decimal('-87.61350386'),
'(41.823879885, -87.613503857)'),
 (3778925, 'HL149610', datetime.datetime(2005, 1, 28, 17, 0), '100XX S WASHTENAW
AVE', '810', 'THEFT', 'OVER \$500', 'STREET', 'FALSE', 'FALSE', 2211, 22, 19, 72,
'6', 1160129, 1838040, 2005, datetime.datetime(2018, 2, 28, 15, 56, 25),
Decimal('41.71128051'), Decimal('-87.68917910'), '(41.711280513,
-87.689179097)'),
 (3324217, 'HK361551', datetime.datetime(2004, 5, 13, 14, 15), '033XX W BELMONT
AVE', '820', 'THEFT', '\$500 AND UNDER', 'SMALL RETAIL STORE', 'FALSE', 'FALSE',
1733, 17, 35, 21, '6', 1153590, 1921084, 2004, datetime.datetime(2018, 2, 28,
15, 56, 25), Decimal('41.93929582'), Decimal('-87.71092344'), '(41.939295821,
-87.710923442)')]
\end{Verbatim}
\end{tcolorbox}
        
    \hypertarget{problem-3}{%
\subsubsection{Problem 3}\label{problem-3}}

\hypertarget{how-many-crimes-involve-an-arrest}{%
\subparagraph{How many crimes involve an
arrest?}\label{how-many-crimes-involve-an-arrest}}

    \begin{tcolorbox}[breakable, size=fbox, boxrule=1pt, pad at break*=1mm,colback=cellbackground, colframe=cellborder]
\prompt{In}{incolor}{8}{\boxspacing}
\begin{Verbatim}[commandchars=\\\{\}]
\PY{o}{\PYZpc{}}\PY{k}{sql} SELECT COUNT(*) as total\PYZus{}arrests FROM CRIME\PYZus{}DATA WHERE arrest=TRUE
\end{Verbatim}
\end{tcolorbox}

    \begin{Verbatim}[commandchars=\\\{\}]
 * ibm\_db\_sa://nrx71347:***@dashdb-txn-sbox-yp-lon02-07.services.eu-
gb.bluemix.net:50000/BLUDB
Done.
    \end{Verbatim}

            \begin{tcolorbox}[breakable, size=fbox, boxrule=.5pt, pad at break*=1mm, opacityfill=0]
\prompt{Out}{outcolor}{8}{\boxspacing}
\begin{Verbatim}[commandchars=\\\{\}]
[(Decimal('163'),)]
\end{Verbatim}
\end{tcolorbox}
        
    \hypertarget{problem-4}{%
\subsubsection{Problem 4}\label{problem-4}}

\hypertarget{which-unique-types-of-crimes-have-been-recorded-at-gas-station-locations}{%
\subparagraph{Which unique types of crimes have been recorded at GAS
STATION
locations?}\label{which-unique-types-of-crimes-have-been-recorded-at-gas-station-locations}}

    \begin{tcolorbox}[breakable, size=fbox, boxrule=1pt, pad at break*=1mm,colback=cellbackground, colframe=cellborder]
\prompt{In}{incolor}{12}{\boxspacing}
\begin{Verbatim}[commandchars=\\\{\}]
\PY{o}{\PYZpc{}}\PY{k}{sql} SELECT DISTINCT(primary\PYZus{}type) as unq\PYZus{}crimes\PYZus{}at\PYZus{}gas\PYZus{}station FROM CRIME\PYZus{}DATA \PYZbs{}
\PY{n}{WHERE} \PY{n}{location\PYZus{}description}\PY{o}{=}\PY{l+s+s1}{\PYZsq{}}\PY{l+s+s1}{GAS STATION}\PY{l+s+s1}{\PYZsq{}}
\end{Verbatim}
\end{tcolorbox}

    \begin{Verbatim}[commandchars=\\\{\}]
 * ibm\_db\_sa://nrx71347:***@dashdb-txn-sbox-yp-lon02-07.services.eu-
gb.bluemix.net:50000/BLUDB
Done.
    \end{Verbatim}

            \begin{tcolorbox}[breakable, size=fbox, boxrule=.5pt, pad at break*=1mm, opacityfill=0]
\prompt{Out}{outcolor}{12}{\boxspacing}
\begin{Verbatim}[commandchars=\\\{\}]
[('CRIMINAL TRESPASS',), ('NARCOTICS',), ('ROBBERY',), ('THEFT',)]
\end{Verbatim}
\end{tcolorbox}
        
    Hint: Which column lists types of crimes e.g.~THEFT?

    \hypertarget{problem-5}{%
\subsubsection{Problem 5}\label{problem-5}}

\hypertarget{in-the-cenus_data-table-list-all-community-areas-whose-names-start-with-the-letter-b.}{%
\subparagraph{In the CENUS\_DATA table list all Community Areas whose
names start with the letter
`B'.}\label{in-the-cenus_data-table-list-all-community-areas-whose-names-start-with-the-letter-b.}}

    \begin{tcolorbox}[breakable, size=fbox, boxrule=1pt, pad at break*=1mm,colback=cellbackground, colframe=cellborder]
\prompt{In}{incolor}{13}{\boxspacing}
\begin{Verbatim}[commandchars=\\\{\}]
\PY{o}{\PYZpc{}}\PY{k}{sql} SELECT COMMUNITY\PYZus{}AREA\PYZus{}NAME FROM CENSUS\PYZus{}DATA WHERE COMMUNITY\PYZus{}AREA\PYZus{}NAME LIKE \PYZsq{}B\PYZpc{}\PYZsq{}
\end{Verbatim}
\end{tcolorbox}

    \begin{Verbatim}[commandchars=\\\{\}]
 * ibm\_db\_sa://nrx71347:***@dashdb-txn-sbox-yp-lon02-07.services.eu-
gb.bluemix.net:50000/BLUDB
Done.
    \end{Verbatim}

            \begin{tcolorbox}[breakable, size=fbox, boxrule=.5pt, pad at break*=1mm, opacityfill=0]
\prompt{Out}{outcolor}{13}{\boxspacing}
\begin{Verbatim}[commandchars=\\\{\}]
[('Belmont Cragin',),
 ('Burnside',),
 ('Brighton Park',),
 ('Bridgeport',),
 ('Beverly',)]
\end{Verbatim}
\end{tcolorbox}
        
    \hypertarget{problem-6}{%
\subsubsection{Problem 6}\label{problem-6}}

\hypertarget{which-schools-in-community-areas-10-to-15-are-healthy-school-certified}{%
\subparagraph{Which schools in Community Areas 10 to 15 are healthy
school
certified?}\label{which-schools-in-community-areas-10-to-15-are-healthy-school-certified}}

    \begin{tcolorbox}[breakable, size=fbox, boxrule=1pt, pad at break*=1mm,colback=cellbackground, colframe=cellborder]
\prompt{In}{incolor}{15}{\boxspacing}
\begin{Verbatim}[commandchars=\\\{\}]
\PY{o}{\PYZpc{}}\PY{k}{sql} SELECT  NAME\PYZus{}OF\PYZus{}SCHOOL FROM SCHOOLS WHERE HEALTHY\PYZus{}SCHOOL\PYZus{}CERTIFIED=\PYZsq{}Yes\PYZsq{} \PYZbs{}
    \PY{n}{AND} \PY{p}{(}\PY{n}{COMMUNITY\PYZus{}AREA\PYZus{}NUMBER} \PY{n}{BETWEEN} \PY{l+m+mi}{10} \PY{n}{AND} \PY{l+m+mi}{15}\PY{p}{)}
\end{Verbatim}
\end{tcolorbox}

    \begin{Verbatim}[commandchars=\\\{\}]
 * ibm\_db\_sa://nrx71347:***@dashdb-txn-sbox-yp-lon02-07.services.eu-
gb.bluemix.net:50000/BLUDB
Done.
    \end{Verbatim}

            \begin{tcolorbox}[breakable, size=fbox, boxrule=.5pt, pad at break*=1mm, opacityfill=0]
\prompt{Out}{outcolor}{15}{\boxspacing}
\begin{Verbatim}[commandchars=\\\{\}]
[('Rufus M Hitch Elementary School',)]
\end{Verbatim}
\end{tcolorbox}
        
    \hypertarget{problem-7}{%
\subsubsection{Problem 7}\label{problem-7}}

\hypertarget{what-is-the-average-school-safety-score}{%
\subparagraph{What is the average school Safety
Score?}\label{what-is-the-average-school-safety-score}}

    \begin{tcolorbox}[breakable, size=fbox, boxrule=1pt, pad at break*=1mm,colback=cellbackground, colframe=cellborder]
\prompt{In}{incolor}{19}{\boxspacing}
\begin{Verbatim}[commandchars=\\\{\}]
\PY{o}{\PYZpc{}}\PY{k}{sql} SELECT AVG(SAFETY\PYZus{}SCORE) as average\PYZus{}school\PYZus{}safety\PYZus{}score FROM SCHOOLS
\end{Verbatim}
\end{tcolorbox}

    \begin{Verbatim}[commandchars=\\\{\}]
 * ibm\_db\_sa://nrx71347:***@dashdb-txn-sbox-yp-lon02-07.services.eu-
gb.bluemix.net:50000/BLUDB
Done.
    \end{Verbatim}

            \begin{tcolorbox}[breakable, size=fbox, boxrule=.5pt, pad at break*=1mm, opacityfill=0]
\prompt{Out}{outcolor}{19}{\boxspacing}
\begin{Verbatim}[commandchars=\\\{\}]
[(Decimal('49.504873'),)]
\end{Verbatim}
\end{tcolorbox}
        
    \hypertarget{problem-8}{%
\subsubsection{Problem 8}\label{problem-8}}

\hypertarget{list-the-top-5-community-areas-by-average-college-enrollment-number-of-students}{%
\subparagraph{List the top 5 Community Areas by average College
Enrollment {[}number of
students{]}}\label{list-the-top-5-community-areas-by-average-college-enrollment-number-of-students}}

    \begin{tcolorbox}[breakable, size=fbox, boxrule=1pt, pad at break*=1mm,colback=cellbackground, colframe=cellborder]
\prompt{In}{incolor}{21}{\boxspacing}
\begin{Verbatim}[commandchars=\\\{\}]
\PY{o}{\PYZpc{}\PYZpc{}}\PY{k}{sql} 
SELECT COMMUNITY\PYZus{}AREA\PYZus{}NAME, AVG(COLLEGE\PYZus{}ENROLLMENT) as \PYZdq{}AVG\PYZus{}ENROLLMENT\PYZdq{}  
FROM SCHOOLS
GROUP BY COMMUNITY\PYZus{}AREA\PYZus{}NAME 
ORDER BY AVG\PYZus{}ENROLLMENT DESC
LIMIT 5
\end{Verbatim}
\end{tcolorbox}

    \begin{Verbatim}[commandchars=\\\{\}]
 * ibm\_db\_sa://nrx71347:***@dashdb-txn-sbox-yp-lon02-07.services.eu-
gb.bluemix.net:50000/BLUDB
Done.
    \end{Verbatim}

            \begin{tcolorbox}[breakable, size=fbox, boxrule=.5pt, pad at break*=1mm, opacityfill=0]
\prompt{Out}{outcolor}{21}{\boxspacing}
\begin{Verbatim}[commandchars=\\\{\}]
[('ARCHER HEIGHTS', Decimal('2411.500000')),
 ('MONTCLARE', Decimal('1317.000000')),
 ('WEST ELSDON', Decimal('1233.333333')),
 ('BRIGHTON PARK', Decimal('1205.875000')),
 ('BELMONT CRAGIN', Decimal('1198.833333'))]
\end{Verbatim}
\end{tcolorbox}
        
    \hypertarget{problem-9}{%
\subsubsection{Problem 9}\label{problem-9}}

\hypertarget{use-a-sub-query-to-determine-which-community-area-has-the-least-value-for-school-safety-score}{%
\subparagraph{Use a sub-query to determine which Community Area has the
least value for school Safety
Score?}\label{use-a-sub-query-to-determine-which-community-area-has-the-least-value-for-school-safety-score}}

    \begin{tcolorbox}[breakable, size=fbox, boxrule=1pt, pad at break*=1mm,colback=cellbackground, colframe=cellborder]
\prompt{In}{incolor}{22}{\boxspacing}
\begin{Verbatim}[commandchars=\\\{\}]
\PY{o}{\PYZpc{}\PYZpc{}}\PY{k}{sql} 
SELECT COMMUNITY\PYZus{}AREA\PYZus{}NAME FROM SCHOOLS
WHERE SAFETY\PYZus{}SCORE IN
(SELECT MIN(SAFETY\PYZus{}SCORE) FROM SCHOOLS)
\end{Verbatim}
\end{tcolorbox}

    \begin{Verbatim}[commandchars=\\\{\}]
 * ibm\_db\_sa://nrx71347:***@dashdb-txn-sbox-yp-lon02-07.services.eu-
gb.bluemix.net:50000/BLUDB
Done.
    \end{Verbatim}

            \begin{tcolorbox}[breakable, size=fbox, boxrule=.5pt, pad at break*=1mm, opacityfill=0]
\prompt{Out}{outcolor}{22}{\boxspacing}
\begin{Verbatim}[commandchars=\\\{\}]
[('WASHINGTON PARK',)]
\end{Verbatim}
\end{tcolorbox}
        
    \hypertarget{problem-10}{%
\subsubsection{Problem 10}\label{problem-10}}

\hypertarget{without-using-an-explicit-join-operator-find-the-per-capita-income-of-the-community-area-which-has-a-school-safety-score-of-1.}{%
\subparagraph{{[}Without using an explicit JOIN operator{]} Find the Per
Capita Income of the Community Area which has a school Safety Score of
1.}\label{without-using-an-explicit-join-operator-find-the-per-capita-income-of-the-community-area-which-has-a-school-safety-score-of-1.}}

    \begin{tcolorbox}[breakable, size=fbox, boxrule=1pt, pad at break*=1mm,colback=cellbackground, colframe=cellborder]
\prompt{In}{incolor}{24}{\boxspacing}
\begin{Verbatim}[commandchars=\\\{\}]
\PY{o}{\PYZpc{}\PYZpc{}}\PY{k}{sql}
SELECT CD.PER\PYZus{}CAPITA\PYZus{}INCOME, CPS.COMMUNITY\PYZus{}AREA\PYZus{}NAME, CPS.SAFETY\PYZus{}SCORE
FROM CENSUS\PYZus{}DATA AS CD, SCHOOLS AS CPS
WHERE CPS.COMMUNITY\PYZus{}AREA\PYZus{}NUMBER=CD.COMMUNITY\PYZus{}AREA\PYZus{}NUMBER
ORDER BY CPS.SAFETY\PYZus{}SCORE 
LIMIT 1
\end{Verbatim}
\end{tcolorbox}

    \begin{Verbatim}[commandchars=\\\{\}]
 * ibm\_db\_sa://nrx71347:***@dashdb-txn-sbox-yp-lon02-07.services.eu-
gb.bluemix.net:50000/BLUDB
Done.
    \end{Verbatim}

            \begin{tcolorbox}[breakable, size=fbox, boxrule=.5pt, pad at break*=1mm, opacityfill=0]
\prompt{Out}{outcolor}{24}{\boxspacing}
\begin{Verbatim}[commandchars=\\\{\}]
[(13785, 'WASHINGTON PARK', 1)]
\end{Verbatim}
\end{tcolorbox}
        
    Copyright © 2018
\href{cognitiveclass.ai?utm_source=bducopyrightlink\&utm_medium=dswb\&utm_campaign=bdu}{cognitiveclass.ai}.
This notebook and its source code are released under the terms of the
\href{https://bigdatauniversity.com/mit-license/}{MIT License}.


    % Add a bibliography block to the postdoc
    
    
    
\end{document}
